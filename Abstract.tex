\documentclass[a4paper]{article}
\usepackage[a4paper]{geometry}
\usepackage{hyperref}


\begin{document}

\title{AnonDroid}
\author{xunil154 }
\date{\today{}}

\maketitle

\vspace{6cm}
\begin{abstract}
AnonDroid is a project designed to make Android users as anonymous to Google as
possible. It will include proxying features, bogus data injections, Google Play
download spoofing and much more.
\end{abstract}

\cleardoublepage
\tableofcontents

\cleardoublepage


\section{Introduction}

  It appears that there is a large market of people who are begining to worry
about how much information Google is able to obtain about their daily lives.
Information such as, contacts, searches, favorite videos, emails, GPS
locations, favorite foods, and the list continues. With the recent explosion of
Android devices on the market it begs the questions of just how much
information is Google now able to collect on a persons? New applications such
as FourSquare are just downright creepy. This is where the AnonDroid software
is going to come into play.

  The primary goal of AnonDroid is to help increase the privacy of the end
users by either proxying end users data through alternate sessions with Google,
generating false information about the end users, or even going as far as to
inject fake searches to throw off potential watchers. By doing so AnonDroid will
effectively allow it's users to stay as anonymous as possible.


\section{Limitations}
  
  Obviously it is impossible to create a fully anonymous Android device
simply due to the nature of the data. The android devices will still have
hardware identifiers which allow them to connect into the cellular networks.
And we will not be able to spoof this ID because it will effectively make the
Android device useless on any cellular network. The other issue arises with 
email. Since uses need to login to their email accounts, there is nothing we can
do to break that link from that account to that end user. In the case of email,
the only piece of information that we could possibly anonymize is the IP
address used to login to Google.

\section{Structure}

AnonDroid will be a complex system consisting of several pieces of software.
There will need to be an Android application that runs on the end device. This
software will most likely will need to have root privileges. The second piece
of software will be ran on a server which will perform the anonymization of the
user data fed to it by the android software.

The Android software will tunnel all non-cellular network traffic to the centralized (for now) server. It will then be the responsibility of the server to
anonymize as much of the traffic as possible before sending it onto Google.

\section{Android software}

\subsection{Overview}

This software will be ran on the end users devices. It will be responsible for
thee main tasks. 

\begin{itemize}
  \item Creating a VPN/SSH tunnel back to the central server
  \item Intercepting all traffic leaving the phone
  \item Anonymizing GPS coordinates 
\end{itemize}

\hspace{-0.65cm}
As well as implementing these tasks, it will need to incorporate these main
features as well

\begin{itemize}
  \item Easily configurable
  \item Completely transparent to the user
  \item One click enable/disable functionality
  \item Generate reports for the user about what personal information is being
    leaked 
\end{itemize}

\subsubsection{VPN}

There are several different VPN/SSH proxy application for the Android platform.
Rather than reinventing the wheel so to speak, we will use a pre-existing
platform for this. The only tricky part will most likely be properly
configuring the routing inside the Android system in order to force all
outgoing traffic to use the new VPN.

\section{Server Software}

The server is going to be the most complex portion of AnonDroid since it will
need to perform several different tasks without incurring too much of a delay for the end user. The server will need to be:

\begin{itemize}
\item Secure
\item Scalable
\item Configurable
\item Modular
\item Able to provide many different tasks to provide anonymization.
  \begin{itemize}
    \item Generate false information
    \item Inject bogus queries
    \item Filter personal information in a secure manner
    \item (future) Virtual android devices
  \end{itemize}

\end{itemize}

\subsection{Security}

Because there will be sensitive user data being sent through this server, it is
\emph{critical} that this server is heavily locked down and monitored. It is
also vital that we develop this application from the ground up with security in
mind. It is my suggestion that we use the following principles when designing the software:

\begin{itemize}

\item Randomize memory before releasing it back to the system
\item Encrypt all data that gets written to disk including logs
\item Ensure that \emph{NO} user data ever touches the disk, or if it does
encrypt it first

\end{itemize}

\subsection{Scalability}
Because it will be possible that the server(s) will be receiving a large
quantity of data, it is vital that they will be able to scale incredibly well.
There are several different models that can assist in this process. One of
which will be to use asynchronous connections and limit the number of threads
which are running. There is also a possibility of using a front proxy such as
Nginx (\href{http://wiki.nginx.org/Main}{nginx.org}).

\subsection{Distribution}
For the proof of concept, everything will be processed on a single server.
However, if the final product is on a single server then it will be easy to
take down that server should Google decide to. 

  Instead what we will need is a cluster of computers in which contributing
members will be able to volunteer their resources and contribute to the
AnonDroid project. This will be the most difficult portion of the AnonDroid
project because we will need to protect user data as it enters and exits these
volunteered computers. There will also be the issue of releasing information
about computers in the cluster to the Android devices as well has. As of right
now a P2P database will probably be the best solution and we can adopt
principles from the BitCoin system as well as utilize the torrent network. 

\subsection{Anonymization}

For the proof of concept, it is suggested that we simply try to anonymize
Google searches that are generated by the Android devices. To do this we can
fork the GoogleSharing project
(\href{http://www.googlesharing.net/}{googlesharing.net}) and customize it for
our purposes and/or build a plugin for Android to use GoogleSharing. Doing the
later would provide several benefits such as contributing back to the
GoogleSharing project, as well as increase the availability of GoogleSharing to
Android users.

We will also be able to make extensive use of the Tor network
\href{https://www.torproject.org/}{torproject.org}. We can also contribute to
this project in the future by requiring that all cluster nodes (see
Distribution section) partake in the Tor network as routing nodes. 


\section{Future}
It is expected that this project will continually grow and incorporate new
technologies into it. Thus we will need to design the core system in such a way
as to make it easy to take data and feed it into a new component which might be
written in a different language. 

\subsection{Virtual devices}
One idea that has been presented is the possibility of creating virtual Android
devices which end users can then get remote desktop to. Afterward the virtual
Android device can then be completely wiped and reused by another user. While
this is a fantastic idea it appears that we are going to need to create our own
remote desktop application for Android devices. After briefly looking, there
are several remote desktop applications designed to connect to desktop
machines, but none to connect to Android devices.


\end{document}
