Anondroid
=========
AnonDroid is a project designed to make Android users as anonymous to Google as
possible. It will include proxying features, bogus data injections, Google Play
download spoofing and much more.  j

<braindump>

<Introduction>

  It appears that there is a large market of people who are begining to worry
about how much information Google is able to obtain about their daily lives.
Information such as, contacts, searches, favorite videos, emails, GPS
locations, favorite foods, and the list continues. With the recent explosion of
Android devices on the market it begs the questions of just how much
information is Google now able to collect on a persons? New applications such
as FourSquare are just downright creepy. This is where the AnonDroid software
is going to come into play.

  The primary goal of AnonDroid is to help increase the privacy of the end
users by either proxying end users data through alternate sessions with Google,
generating false information about the end users, or even going as far as to
inject fake searches to throw off potential watchers. By doing so AnonDroid will
effectively allow it's users to stay as anonymous as possible.

</Introduction>

<Limitations>
  
  Obviousally it is impossible to create a fully anonymous Android device
simply due to the nature of the data. The android devices will still have
hardware identifiers which allow them to connect into the cellular networks.
And we will not be able to spoof this ID because it will effectively make the
Android device useless on any cellular network. The other issue arises with 
email. Since uses need to login to their email accounts, there is nothing we can
do to break that link from that account to that end user. In the case of email,
the only piece of information that we could possibly anonymize is the IP
address used to login to Google.

</Limitations>

</braindump>
